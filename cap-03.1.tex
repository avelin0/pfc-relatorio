\chapter{Algoritmo}\label{cap:algoritmo}

O algoritmo proposto nessa pesquisa é baseado no algoritmo "\textit{Watershed Algorithm Based On Connected Components}", apresentado em \cite{ruparelia2012implementation}, o qual trata-se de uma variação da técnica de \textit{watershed} com resultados de segmentação considerados satisfatórios além da menor complexidade computacional com relação à abordagem tradicional \textit{watershed}.

\section{\textit{Watershed}}\label{sec:watershed}
\textit{Watershed} é uma técnica de segmentação eficiente e poderosa. Tal técnica tem como vantagem gerar sempre resultados com contornos fechados e bem definidos, o que é de grande importância para o processo de segmentação de imagens. Além disso, comparada a outras técnicas de segmentação, apresenta menor complexidade computacional.

Duas abordagens são utilizadas para explicar a ideia básica do \textit{watershed} na segmentação de imagens. 
A primeira, denominada "\textit{flooding based watershed}", trata a imagem em níveis de cinza com uma paisagem formada por vales, onde encontram-se os mínimos locais. Considerando um processo de inundação com a água subindo a partir de cada um dos vales, serão construídas barragens nos pontos de encontro da água oriunda de dois vales distintos, chamadas de \textit{watersheds}. Essas barragens, portanto, são interpretadas como bordas entra diferentes regiões da imagem.\citep{l6} 

% Figura
	\begin{figure}[!htb]
       \begin{center}  
          \includegraphics[width=0.6\columnwidth]{img/abordagem_flooding.jpg}
           \caption{\label{fig:abordagem_flooding}Abordagem "\textit{flooding based watershed}".\cite{regseg1}}
           % \vspace{2.0em}
       \end{center}
   \end{figure} 


A outra abordagem, denominada "\textit{rainfalling based watershed}" trata a imagem em níveis de cinza da mesma forma que a primeira, porém o fluxo de água ocorre a partir de gotas de água que ao incidirem em qualquer ponto da superfície escorrerão para um determinado vale, onde encontra-se um mínimo local. O conjunto de pontos para os quais a gota de água escorre para o mesmo local é interpretada como uma região e os limites entre duas regiões adjacentes,  interpretados como bordas, são as \textit{watersheds}. \citep{l6} 

% Figura 
	\begin{figure}[!htb]
       \begin{center}  
          \includegraphics[width=0.6\columnwidth]{img/abordagem_rainfalling.jpg}
           \caption{\label{fig:abordagem_rainfalling}Abordagem "\textit{rainfalling based watershed}". \cite{ruparelia2012implementation}}
           % \vspace{2.0em}
       \end{center}
   \end{figure} 
	

Ambas abordagens tratam da mesma ideia básica da técnica, sendo duas formas diferentes de ilustrar seus funcionamento sobre os quais diferentes algoritmos são propostos.

Um dos principais problemas relacionados à abordagem tradicional do \textit{watershed} é o problema de \textit{over-segmentation}. Para reduzir esse problema, diversas variações de algoritmos baseados na técnica de \textit{watershed} foram implementados, tal qual o algoritmo utilizado nessa pesquisa que será explicado detalhadamente na seção \ref{sec:alg} do capítulo \ref{cap:algoritmo}. Assim as diferenças entre o algoritmo proposto e o tradicional, como o implementado na biblioteca OpenCV, apresentada na seção \ref{sec:opencv} do capítulo \ref{cap:ferramentas}, conduzirão a uma análise comparativa entre os resultados obtidos por meio de cada um deles.

\section{\textit{Watershed Algorithm Based On Connected Components}}
Destacam-se as seguintes características entre este algoritmo e a abordagem tradicional \textit{watershed} que demonstram sua maior eficiência:

% \section{Características}
\begin{itemize}
    \item Fila FIFO ao invés de fila hierárquica - algoritmo tradicional - que requer sequências de acesso à memória não uniformes;
    \item Estrutura de dados mais simples; e
    \item Menor tempo de execução.
\end{itemize}    

Este algoritmo tem como princípio conectar cada \textit{pixel} (componente), caso este não seja o mínimo local, ao menor \textit{pixel} vizinho. Todos os \textit{pixels} direcionados para o mesmo mínimo local formam um segmento e são, portanto, rotulados da mesma forma.

% Figura
	\begin{figure}[!htb]
       \begin{center}  
          \includegraphics[width=0.9\columnwidth]{img/connected_components.jpg}
           \caption{\label{fig:connected_components}Ilustração do funcionamento da técnica \textit{Watershed Based On Connected Components}. \cite{ruparelia2012implementation}}
           % \vspace{2.0em}
       \end{center}
   \end{figure}

\section{Implementação do Algoritmo}\label{sec:alg}

O algoritmo de segmentação proposto nessa pesquisa, assim como os demais algoritmos baseados na técnica de segmentação por \textit{watershed}, segue a estrutura apresentada na figura \ref{fig:diagrama_blocos_algoritmo}. A imagem original passa por uma etapa de pré-processamento, em que diferentes procedimentos são realizados. Após a etapa de pré-processamento, a imagem é segmentada pela técnica \textit{watershed} e seu resultado pode ser processado, na etapa de pós-processamento, a fim de corrigir algumas falhas geradas pelo processo de segmentação para, finalmente, obter a segmentação como resultado final.
Todas as etapas serão explicadas nas próximas seções de acordo como foram implementadas no algoritmo proposto por essa pesquisa.

% Figura
	\begin{figure}[!htb]
       \begin{center}  
          \includegraphics[width=0.7\columnwidth]{img/diagrama_blocos_algoritmo.jpg}
           \caption{\label{fig:diagrama_blocos_algoritmo}Diagrama de blocos do algoritmo de segmentação dessa pesquisa.}
           % \vspace{2.0em}
       \end{center}
   \end{figure} 

\subsection{Pré-Processamento}
A etapa de pré-processamento tem como objetivo preparar a imagem que será utilizada na segmentação efetivamente. Para isso alguns procedimentos são realizados com o objetivo de retirar ruídos e realçar bordas para facilitar o processo de segmentação a fim de obter um resultado mais eficiente. Além disso, assim como a etapa de pós-processamento, esta etapa visa diminuir o problema de \textit{over-segmentation}, o qual tende a ocorrer quando aplica-se a técnica de \textit{watershed}.
A ordem em que esses procedimentos ocorrem nesta etapa consta na estrutura apresentada pela figura \ref{fig:diagrama_blocos_preprocessamento}.

% Figura
	\begin{figure}[!htb]
       \begin{center}  
          \includegraphics[width=0.8\columnwidth]{img/diagrama_blocos_preprocessamento.jpg}
           \caption{\label{fig:diagrama_blocos_preprocessamento}Diagrama de blocos da etapa de pré-processamento.}
           % \vspace{2.0em}
       \end{center}
   \end{figure}

\subsubsection{Conversão para Cinza}
Um dos procedimentos realizados na etapa de pré-processamento é a conversão da imagem original colorida para a imagem em níveis de cinza. Utiliza-se a quantização em 256 níveis de cinza (8 bits), onde o nível zero representa o preto e o nível 255, o branco.
O objetivo dessa conversão é tratar a imagem em um canal já que o escopo da presente pesquisa é tratar imagens em escala de cinza.

\subsubsection{Filtro Bilateral}
O filtro biltateral é utilizado como um dos primeiros procedimentos da etapa de pré-processamento a fim de reduzir os ruídos presentes na imagem a ser segmentada, preservando suas bordas.

Abaixo a definição matemática do Filtro Bilateral:

% Figura
	\begin{figure}[!htb]
       \begin{center}  
          \includegraphics[width=0.7\columnwidth]{img/definicao_matematica_filtro_bilateral.jpg}
           \caption{\label{fig:definicao_matematica_filtro_bilateral}Definição matemática do Filtro Bilateral.}
           % \vspace{2.0em}
       \end{center}
   \end{figure}
   
%\subsubsection{Operador \textit{Not}}
%O operador not é utilizado no pré-processamento para que o algoritmo \textit{watersheed } seja executado da maneira correta, isto é, dos plateaus para os mínimos locais através. Ele inverte cada bit de um \textit{array}.

\subsubsection{Operador Sobel}
Operador de detecção de bordas, que se baseia no operador gradiente. Esse operador é utilizado sobre a imagem resultante do operador \textit{not} com a finalidade de gerar uma imagem com as bordas detectadas brilhantes em um fundo (\textit{background}) mais escuro.

Os gradientes horizontal, \textit{Gx}, e vertical, \textit{Gy}, são calculados considerando uma janela de tamanho 3, isto é, \textit{3 x 3 pixels}, e calcula-se uma aproximação do gradiente em cada ponto da seguinte forma: \citep{sobel}

Abaixo a definição matemática do Operador Sobel:
% Figura
	\begin{figure}[!htb]
       \begin{center}  
          \includegraphics[width=0.7\columnwidth]{img/definicao_matematica_sobel.jpg}
           \caption{\label{fig:definicao_matematica_sobel}Definição matemática do Operador Sobel.}
           % \vspace{2.0em}
       \end{center}
   \end{figure} 
% Equação
%\[G = |G_{x}| + |G_{y}|\]

% Código

%\subsubsection{Filtro de Mediana}
%O objetivo do filtro de mediana é a redução ou, até mesmo, a remoção de ruídos das imagens, a fim de suavizá-las e torná-las mais tratáveis para o processo de segmentação. O filtro de mediana é eficaz para o tratamento de ruídos impulsivos, como o ruído Gaussiano (aleatório) e, principalmente, o ruído sal e pimenta, em que os \textit{pixels} ruidosos assumem os valores máximos e mínimos da imagem.
%Para a implementação do filtro de mediana, considera-se uma imagem com \textit{n x m pixels} e um filtro com janela de \textit{k x k pixels}, onde \textit{k < n e k < m}. Em casa janela considerada na imagem, o valor de cada \textit{pixel} é substituído pelo valor da mediana da mesma janela. No algoritmo, usa-se \textit{k = 3} e o funcionamento do filtro pode ser entendido na figura \ref{fig:filtromediana}.

% Figura
	%\begin{figure}[!htb]
     %  \begin{center}  
      %    \includegraphics[width=0.8\columnwidth]{img/filtromediana.jpg}
       %    \caption{\label{fig:filtromediana}Ilustração do funcionamento do filtro de mediana. \cite{ruparelia2012implementation}}
           % \vspace{2.0em}
       %\end{center}
  % \end{figure}
   
% Código

\subsubsection{Operadores Morfológicos}
Após a aplicação de um detector de bordas, como o operador Sobel, a imagem pode ficar com bordas "falhadas". Como a técnica de \textit{watershed} requer que as regiões estejam "fechadas", ou seja, sem falhas nas bordas, é aplicado um conjunto de operações morfológicas para conectar estas regiões fragmentadas. 
Existem duas operações morfológicas básicas:
% \section{Operações morfológicas básicas}
\begin{itemize}
    \item Erosão; e
    \item Dilatação.
\end{itemize}    

A biblioteca OpenCV fornece 5 transformações morfológicas baseadas nessas duas operações básicas:
% \section{Transformações morfológicas}
\begin{itemize}
    \item \textit{Opening};
    \item \textit{Closing};
    \item \textit{Morphological Gradient};
    \item \textit{Top Hat}; e 
    \item \textit{Black Hat}.
\end{itemize} 

No algoritmo utiliza-se apenas a transformação do gradiente morfológico - \textit{Morphological Gradient} -, a qual é a diferença entre a dilatação e a erosão de uma imagem. Essa transformação é útil para encontrar o contorno de um objeto em uma dada imagem. \citep{morphology_transformations}
% Equação
\[dst = morph_{grad}( src, element ) = dilate( src, element ) - erode( src, element )\]

\subsection{Segmentação por Técnica Baseada em \textit{Watershed}}
Dentre as duas principais abordagens da técnica de \textit{watershed} mencionadas na seção \ref{sec:watershed} do capítulo \ref{cap:algoritmo}, optou-se pela abordagem "\textit{rainfalling based watershed}" a fim de servir como base para a implementação da etapa de segmentação implementada neste algoritmo. Essa etapa é composta por três passos (\textit{steps}). 

\subsubsection{\textit{Step} 1}
% Explicar 
O objetivo do passo 1 é encontrar os mínimos locais na imagem. Inicialmente, o \textit{array} v[p] e percorre-se a imagem de cima à esquerda até abaixo à direita e, v[p] assumirá o valor zero se o valor de seu vizinho for menos ou igual ao seu, e o valor 1 caso contrário.


% Código 
\begin{algorithm}[H]
\SetAlgoLined

    \SetKwInOut{Input}{Input}
    \SetKwInOut{Output}{Output}

    \underline{STEP1} $(p)$\;
		\If{v[p] != 1}{
  		\For{cada n vizinho de p}{
   		\If{f[n] < f(p)}{
	   		v[p] = 1\;   		
   		}
   }
   }
 
\caption{Algoritmo para o \textit{step} 1 da segmentação \textit{watershed}. \cite{ruparelia2012implementation}}
\end{algorithm}

\subsubsection{\textit{Step} 2} 
% Explicar 
O fundamento do passo 2 é que se um \textit{pixel} está no \textit{plateau} e seu vizinho apontado para um dos mínimos locais, então o \textit{pixel} aponta para o respectivo vizinho. Para isso, considera-se os \textit{pixels} com v[p] diferente de 1 e com seus \textit{pixels} vizinhos no mesmo \textit{plateau} com v[p]=1, ou seja, regiões que não são de mínimo local. Em seguida, calcula-se a menor distancia até um mínimo local.

% Código 
\begin{algorithm}[H]
\SetAlgoLined

    \SetKwInOut{Input}{Input}
    \SetKwInOut{Output}{Output}

    \underline{STEP2} $(p)$\;
		
		  \If{v[p] != 1}{
  		min = VMAX, para cada n de p\\				
   		\If{f(n) = f(p) {\bf and} v[n] > 0 {\bf and} v[n] < min}{
	   		min = v[n]\;   		
   		}
   		\If{min != VMAX {\bf and} v[p] != (min+1)  }{
	   		v[p] = min+1\;   		
   		}
   
   }
 
\caption{Algoritmo para o \textit{step} 2 da segmentação \textit{watershed}. \cite{ruparelia2012implementation}}
\end{algorithm}


\subsubsection{\textit{Step} 3}
% Explicar
O objetivo da da terceira etapa do algoritmo é separar os \textit{pixels} em regiões. Para isso, inicializa-se todos os \textit{pixels} com valor zero. Inicialmente, começa-se a definir as regiões a partir dos mínimos locais cujo v[p]=0 cujos \textit{pixels} vizinhos com o mesmo valor na escala de cinza ainda não estão associados a uma região definida. Essas regiões são propagadas para seus \textit{pixels} vizinhos de acordo com o valor de v[p] para criar regiões cujo centro é um mínimo local. Em seguida, regiões similares são criadas para todos os mínimos locais por este mesmo procedimento.

% Código 
\begin{algorithm}[H]
\SetAlgoLined

    \SetKwInOut{Input}{Input}
    \SetKwInOut{Output}{Output}

    \underline{STEP3} $(p)$\;
		
		lmin=LMAX, fmin=f(p)\\
   \uIf{v[p] = 0}{
        \For{cada n vizinho de p}{
            \If{f(n) = f(p) {\bf and} l[n] > 0  {\bf and} l[n] < lmin}{
                lmin = l[n]
            }
            \If{lmin = LMAX  {\bf and} l[p] = 0 }{
                lmin = New\_label+1
            }
        }
    }
    
    \uElseIf{v[p] = 1}{
       \For{cada n vizinho de p}{
            \If{f(n) < fmin}{
                fmin = f[n]
            }
        }
        \For{cada n vizinho de p}{
            \If{f(n) = fmin {\bf and} l[n] > 0 {\bf and} l[n] < lmin}{
                lmin = l[n]
            }
        }
       
    }   
    \Else{
        \For{cada n vizinho de p}{
            \If{f(n) = f(p) {\bf and} v[n] = v[p]-1 {\bf and} l[n] > 0 {\bf and} l[n] < lmin}{
                lmin = l[n]
            }
        } 
    }
    
    \If{lmin != LMAX {\bf and} l[n] != LMIN}{
        l[p] = lmin
    }
 
 
\caption{Pseudo código para o \textit{step} 3 da segmentação \textit{watershed}.\cite{ruparelia2012implementation}}
\end{algorithm}




\subsection{Pós-Processamento}
Após as etapas de pré-processamento e segmentação, ainda podem ser necessárias algumas correções relacionadas ao problema de \textit{over-segmentation}. É comum obter diferentes segmentos que fazem parte de uma mesma região como resultado do processo de segmentação. Para tanto pode-se utilizar algum método de  agrupamento para mesclar esses segmentos a fim de melhorar a qualidade da segmentação.