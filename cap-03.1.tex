\chapter{Algoritmo}

O algoritmo proposto nessa pesquisa é baseado no algoritmo "\textit{Watershed Algorithm Based On Connected Components}", apresentado na tese xxxx, o qual trata-se de uma variação da técnica de \textit{watershed} com resultados de segmentação considerados bastante satisfatórios além da menor complexidade computacional com relação à abordagem tradicional \textit{watershed}.

\section{\textit{Watershed}}

\textit{Watershed} é uma técnica de segmentação criada em xxxx, Trata-se de uma ferramenta bastante eficiente e poderosa. Tal técnica tem como vantagem gerar sempre resultados com contornos fechados e bem definidos, o que é de grande importância para o processo de segmentação de imagens. Além disso, comparada a outras técnicas de segmentação, apresenta menor complexidade computacional.

Duas abordagens são bastante utilizadas para explicar a ideia básica do \textit{watershed} na segmentação de imagens. 
A primeira, denominada "\textit{flooding based watershed}", trata a imagem em níveis de cinza com uma paisagem formada por vales, onde encontram-se os mínimos locais. Considerando um processo de inundação com a água subindo a partir de cada um dos vales, serão construídas barragens nos pontos de encontro da água oriunda de dois vales distintos, chamadas de \textit{watersheds}. Essas barragens, portanto, são interpretadas como bordas entra diferentes regiões da imagem. 

% Figura 


A outra abordagem, denominada "\textit{rainfalling based watershed}" trata a imagem em níveis de cinza da mesma forma que a primeira, porém o fluxo de água ocorre a partir de gotas de água que ao incidirem em qualquer ponto da superfície escorrerão para um determinado vale, onde encontra-se um mínimo local. O conjunto de pontos para os quais a gota de água escorre para o mesmo local é interpretada como uma região e os limites entre duas regiões adjacentes,  interpretados como bordas, são as \textit{watersheds}.

% Figura 


Ambas abordagens tratam da mesma ideia básica por trás da técnica, sendo duas formas diferentes de ilustrar seus funcionamento sobre os quais diferentes algoritmos são propostos.

Um dos principais problemas relacionados à abordagem tradicional do \textit{watershed} é o problema de \textit{over-segmentation}. Para reduzir esse problema, diversas variações de algoritmos baseados na técnica de \textit{watershed} foram implementados, tal qual o algoritmo utilizado nessa pesquisa que será explicado detalhadamente na seção xx. Assim as diferenças entre o algoritmo proposto e o tradicional, como o implementado na biblioteca OpenCV, apresentada na seção \ref{sec:opencv} do capítulo \ref{cap:ferramentas}, conduzirão a uma análise comparativa entre os resultados obtidos por meio de cada um deles.

\section{\textit{Watershed Algorithm Based On Connected Components}}

Destacam-se as seguintes características entre este algoritmo e a abordagem tradicional \textit{watershed} que demonstram sua maior eficiência:

% \section{Características}
\begin{itemize}
    \item Fila FIFO ao invés de fila hierárquica - algoritmo tradicional - que requer sequências de acesso à memória não uniformes;
    \item Estrutura de dados mais simples; e
    \item Menor tempo de execução.
\end{itemize}    

Este algoritmo tem como princípio conectar cada \textit{pixel} (componente), caso este não seja o mínimo local, ao menor \textit{pixel} vizinho. Todos os \textit{pixels} direcionados para o mesmo mínimo local formam um segmento e são, portanto, rotulados da mesma forma.

% Figura


Os algoritmos baseados na técnica de segmentação por \textit{watershed} seguem a estrutura apresentada na figura xxxx. A imagem original passa por uma etapa de pré-processamento, em que diferentes procedimentos são realizados. Após a etapa de pré-processamento, a imagem é segmentada pela técnica \textit{watershed} e seu resultado pode ser processado, na etapa de pós-processamento, a fim de corrigir algumas falhas geradas pelo processo de segmentação para, finalmente, obter a segmentação como resultado final.
Todas as etapas serão explicadas nas próximas seções de acordo como foram implementadas no algoritmo proposto por essa pesquisa.

% Figura


\section{Pré-Processamento}

 

