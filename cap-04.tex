\chapter{Ferramentas}\label{cap:ferramentas}

\section{Android Studio}
Plataforma de desenvolvimento de aplicativos para sistemas Android. Criado especificamente para esse propósito, a IDE oficial do Android oferece diversos recursos que aceleram o desenvolvimento e aumentam a qualidade dos aplicativos criados.

Tais recursos como ferramentas de edição, depuração, testes e geração de perfis de código motivaram a escolha dessa plataforma para o desenvolvimento da aplicação dessa pesquisa. Além da programação na linguagem Java, é permitido o desenvolvimento em linguagem nativa, C/C++, por meio da ferramenta Android NDK, que será utilizada na implementação dos algoritmos de segmentação presentes nessa aplicação\citep{androidstudio}.

\section{OpenCV}\label{sec:opencv}
Biblioteca \textit{open source} para o desenvolvimento de aplicativos na área de Visão Computacional. Conta com mais de 2500 algoritmos otimizados com abordagens clássicas e no estado da arte nas áreas de visão computacional e aprendizado de máquina, sendo, portanto, uma ferramenta amplamente usada em aplicações que envolvem o processamento de imagens (mais de 14 milhões de \textit{downloads}). Tem interfaces para linguagens como C++, C, Python e Java e suporta diferentes plataformas como Windows, Linux, Mac OS e Android. A maior performance é obtida com seu uso em C++ pelo fato de ser sua linguagem nativa.\citep{opencv}

\section{Android NDK}
O Android Native Developmentt Kit (NDK) é um conjunto de ferramentas que permite a codificação de arquivos de projeto em linguagens nativas como C/C++. Essa ferramenta auxiliará na implementação dos algoritmos de segmentação de imagens que farão parte de nossa aplicação.\citep{ndk}