\chapter{Aplicação}\label{cap:aplicacao}

O objetivo da aplicação como descrito na seção \ref{sec:objetivos} do capítulo \ref{cap:introducao} é realizar a segmentação de imagens por meio de um aplicativo móvel na plataforma Android.
O processo de segmentação é realizado pelo algoritmo de segmentação por técnica baseada em \textit{watershed}, descrito no capítulo \ref{cap:algoritmo}, e a imagem segmentada resultante é apresentada na tela do aplicativo.

\section{Aquisição da imagem}\label{sec:aquisicao_aplicacao}

A aquisição da imagem pode ser feita de duas maneiras diferentes: 

% \section{Aquisição da imagem}
\begin{itemize}
    \item Por meio do acesso à galeria de imagens do dispositivo; e
    \item Por meio da captura de uma imagem através da câmera do dispositivo.
\end{itemize}

\subsection{Galeria de imagens}

Para escolher uma imagem a ser segmentada, basta clicar no botão "galeria" a fim de acessar as imagens existentes no dispositivo e, depois, clicar sobre a imagem desejada.
A imagem escolhida será apresentada na tela, substituindo a imagem anterior.

\subsection{Câmera}

Uma nova imagem pode ser adquirida a qualquer momento para ser utilizada na segmentação por meio do acesso à câmera do dispositivo. Para acessar a câmera, basta clicar no botão "câmera" apresentado na tela inicial da aplicação.

\section{Segmentação da imagem}\label{sec:segmentacao_aplicacao}

A segmentação da imagem, por sua vez, pode ser feita também de duas maneiras diferentes:
% \section{Segmentação da imagem}
\begin{itemize}
    \item Por meio do algoritmo desenvolvido para esta alicação; e
    \item Por meio do algoritmo de \textit{watershed} implementado no OpenCV Java.
\end{itemize}

\subsection{\textit{Watershed} Desenvolvido}

Após a escolha da imagem, ao clicar sobre o botão "\textit{watershed}", a imagem escolhida será segmentada de acordo com o algoritmo desenvolvido neste projeto. A sequência de \textit{steps} apresentados na seção \ref{sec:alg} do capítulo \ref{cap:algoritmo} é executada pelo processador do dispositivo utilizado e o resultado apresentado da imagem segmentada apresentada na tela.  

\subsection{\textit{Watershed} OpenCV Java}

Da mesma forma, com a imagem escolhida, pode-se optar por segmentá-la através do algoritmo já existente e implementado na interface Java da biblioteca OpenCV. Para isso basta clicar sobre o botão "\textit{watershed} OpenCV" e a imagem resultante da segmentação será apresentada na tela do dispositivo. Enquanto que a imagem resultante pelo processo de segmentação que utiliza o algoritmo desenvolvido para a aplicação é colorida, a imagem resultante da segmentação fornecida pelo OpenCV Java é em níveis de cinza, uma vez que foi assim foi implementada.




O objetivo de se colocar a segmentação  \textit{watershed} fornecida pelo OpenCV Java é possibilitar uma análise comparativa dos resultados obtidos.