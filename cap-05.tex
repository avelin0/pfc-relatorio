\chapter{Cronograma}
\section{Definição de Etapas}
\subsection{Escolha do Tema e Estudo de Viabilidade}
A escolha do tema foi a primeira etapa do projeto. O uso de dispositivos móveis, bem como de suas respectivas câmeras tem sido cada dia mais frequentes. Essas câmeras captam informações do ambiente que os cercam e a segmentação de imagem pode ser usado no processamento de imagens de forma a desenvolver soluções computacionalmente automatizáveis.

Após a escolha do tema, foi feito o estudo de viabilidade, de forma a se verificar a possibilidade da cumprimento do objetivo do tema em tempo aceitável.

\subsection{Revisão Bibliográfica}
Referências tais como livros, artigos e outras referências foram estudadas durante o período de revisão bibliográfica, de forma a permitir um embasamento bibliográfico do projeto.

\subsection{Elaboração da Monografia}
Esta fase do projeto se estende até o termino do projeto de fim de curso. Nesta última, confecciona-se um relatório utilizando-se os conhecimentos adquiridos desde o início do projeto.

\subsection{Estudo e Análise dos Algoritmos de Segmentação de Imagem}
Na aplicação proposta, alguns algoritmos de segmentação de imagem serão implementados no dispositivo móvel, de modo a estudá-los e verificar o mais apropriado à aplicação.

\subsection{Implementação}
A implementação ocorrerá após o estudo dos algoritmos e da definição de quais deles são mais indicados à proposta. Em uma primeira fase os algoritmos serão implementados em outros ambientes de desenvolvimento e em uma segunda etapa será realizada a portabilidade para o ambiente Android.

\subsection{Teste}
Os testes com as imagens em diferentes algoritmos serão realizados e, com a implementação pronta, pode-se observar e comparar os resultados dos algoritmos.

\subsection{Entrega do Relatório Final e Apresentação}
A partir das análises e de todo conteúdo, implementação, análises e testes, será feito o relatório e a apresentação à banca.


\section{Entregáveis}
Os entregáveis serão:

\begin{itemize}
\item Projeto do  Aplicativo
\item Escolha dos Algoritmos 
\item Aplicativo em Ambiente Android de Segmentação de Imagens implementado
\item Testes e análises comparativas
\end{itemize}

%Os dois primeiros ítens serão entregues durante a apresentação da Verficação Corrente (VC),e os outros serão entregues por ocasião da apresentação da Verificação Final (VF).

\begin{table}[!htpb]
\centering

% definindo o tamanho da fonte para small
% outros possíveis tamanhos: footnotesize, scriptsize
\begin{small} 
  
% redefinindo o espaçamento das colunas
\setlength{\tabcolsep}{3pt} 

% \cline é semelhante ao \hline, porém é possível indicar as colunas que terão essa a linha horizontal
% \multicolumn{10}{c|}{Meses} indica que dez colunas serão mescladas e a palavra Meses estará centralizada dentro delas.

\begin{tabular}{|c|c|c|c|c|c|c|c|c|c|c|}\hline
 & \multicolumn{9}{c|}{Meses}\\ \cline{2-10}
\raisebox{1.5ex}{Etapa} & FEV & MAR & ABR & MAI & JUN & JUL & AGO & SET & OUT \\ \hline

Escolha do Tema e Estudo de Viabilidade & X & X & X & & & & & &   \\ \hline
Revisão Bibliográfica & X & X & X & X & & & & &   \\ \hline
Elaboração da Monografia & X & X & X & X & X & X & X & X & X    \\ \hline
Estudo e Análise dos Algoritmos  & &  &  & &  & X & X & X &  \\ 
de Segmentação de Imagem & &  &  & &  &  &  &  &  \\ \hline
Implementação & &  &  & &  & X & X & X &    \\ \hline
Teste & & & & & & & & X &   \\ \hline
Entrega do Relatório Final e Apresentação & & & & & & & & & X  \\ \hline

\end{tabular} 
\end{small}
\caption{Cronograma das atividades previstas}
\label{t_cronograma}
\end{table} 