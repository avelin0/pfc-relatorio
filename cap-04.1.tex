\chapter{Conclusão}\label{cap:conclusao}

O projeto atingiu o objtivo principal de desenvolver uma aplicação móvel em ambiente Android para a segmentação de imagens. Todo o estudo sobre o processamento digital de imagens e as diversas técnicas de segmentação formaram a base do conhecimento empregada na concepção desse produto. Além disso pôde-se perceber a importância crescente de sistemas de visão computacional no tratamento de imagens a medida que se busca aproximar da visão e da análise humana bem como suas aplicações.

Como trata-se de um problema mal colocado, isto é, sem solução universal, foi implementado uma técnica de segmentação baseada no algoritmo \textit{watershed} a partir da qual foi possível obter resultados bastante satisfatórios. Certamente,o desenvolvimento de novas técnicas de segmentação e suas implementações tornarão a aplicação mais completa, além de permitir uma análise comparativa dos resultados obtidos, levando a uma ferramenta mais útil para novos projetos de pesquisa na área da visão computacional. Para isso o aplicativo desenvolvido está disponível na plataforma do GitHub, no \textit{link} https://github.com/brunoavelino/SEG, a fim de permitir a colcaboração de estudantes e pesquisadores interessados. A possível integração dessa aplicação com sistemas de manipulação de imagens que necessitam da segmentação como etapa de seus processos é também outra forma de criação de valor por meio deste projeto.

O aprendizado obtido nas etapas de desenvolvimento desse projeto e a criação de um produto com o potencial de impactar positivamente a comunidade científica superaram as expectativas iniciais e satisfizeram seus objetivos.


