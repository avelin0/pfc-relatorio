\chapter{Algoritmos de Segmentação}\label{cap:algoritmos}

Este capitulo descreve 4 técnicas de segmentação de imagens, sendo a por limiar (Thresholding) a mais simples. As técnicas baseadas em bordas e regiões são mais complexas e demandam um ônus computacional mais elevado.

% \section{Tipos de Algoritmos}
\begin{itemize}
    \item Thresholding
    \item Método Baseado em Bordas 
    \item Método Baseado em Regiões
    \begin{itemize}  
        \item Split and Merge
        \item Watershed
    \end{itemize}
\end{itemize}

\subsection{Thresholding}
É o método mais simples de segmentação de imagens.
Busca dividir a imagem em duas categorias: objetos (foreground) e plano de fundo (background). 
Cada pixel é alocado a uma categoria de acordo com seu valor em níveis de cinza.

Dado um limiar T (threshold), o pixel com valor $f_i_j$  e localizado na posição (i,j)  é alocado à:
\\
\begin{cases}
$  categoria 1, se: f_{ij} $\leq$ T. $ \\
$  categoria 2, caso contrário. $ \\
\end{cases}
\\
O limiar T pode ser escolhido manualmente, tentando diferentes valores de T e analisando qual deles é mais eficiente na identificação dos objetos de interesse.
O threshold T também pode ser escolhido a partir do Histograma da imagem, e escolhe-se T como o valor entre as duas distribuições de cinza.

A seguir há a figura \ref{fig:einstein}, uma imagem original do Einstein,  e as duas formas de segmentação de imagens por limiar (thresholding). Enquanto na figura \ref{fig:einsteinglobal} há um limiar T fixo , na  figura \ref{fig:einsteinlocal} ele é variável.

\subsection*{Threshold Global}
O mesmo valor de T é usado para a imagem inteira.

% Figura 
  \begin{figure}[!htb]
       \begin{center}  
          \includegraphics[width=0.3\columnwidth]{img/einstein.jpg}
           \caption{\label{fig:einstein}Imagem original\citep{stanford}.}
           % \vspace{2.0em}
       \end{center}
   \end{figure}

% Figura 
  \begin{figure}[!htb]
       \begin{center}  
          \includegraphics[width=0.3\columnwidth]{img/einstein-globalthresholding127.jpg}
           \caption{\label{fig:einsteinglobal}Imagem segmentada pelo algoritmo threshold global.}
           % \vspace{2.0em}
       \end{center}
   \end{figure}

\subsection*{Threshold Local (ou dinâmico)}
Divide-se a imagem em regiões distintas e adota-se um valor T para cada uma delas, onde esse valor funcionará como threshold local. 

% Figura 
  \begin{figure}[!htb]
       \begin{center}  
          \includegraphics[width=0.3\columnwidth]{img/einstein-localthresholding-adaptivegausian.jpg}
           \caption{\label{fig:einsteinlocal}Imagem segmentada pelo método de segmentação thresholding local, chamado de adaptive gaussian  thresholding.}
           % \vspace{2.0em}
       \end{center}
   \end{figure}
   
   Pode-se observar pela figura \ref{fig:einsteinlocal} que houve a separação em mais regiões nesse método do que na figura \ref{fig:einsteinglobal}, já que foi usado um limiar mais apropriado a cada região.

\subsection{Baseado em Bordas}
Primeiramente, classifica-se os pixels como “borda” ou “não-borda”.
Depois, divide-se a imagem em regiões, baseado nas bordas detectadas.

As bordas são identificadas por meio das descontinuidades, isto é, variações abruptas nos valores dos pixels. 

Como pode-se perceber na figura \ref{fig:smandrill}, por esse método houve a segmentação da imagem de um macaco em regiões de olhos, nariz e outras partes. 

% ------------------------------------------------------------------------------------------------------------
% Figura 
\begin{figure}[!htb]
 \centering
 \def\baselinestretch{1}\small\normalsize
 \includegraphics[width=0.4\textwidth]{img/stf-smandrill.jpg}\qquad
 \includegraphics[width=0.4\textwidth]{img/stf-smandrill-edgedetect.jpg} 
 \caption{\label{fig:smandrill} Uma imagem de um macaco \citep{stanford} à esquerda e à direita segmentada pelo método de detecção de bordas em linguagem de programação python.}
 %\vspace{2.0em}
\end{figure}


\subsection{Baseado em Regiões}
\subsubsection*{O que é uma região?}
Uma região pode ser “definida” como um grupo de pixels conectados com propriedades similares.

Porém, é um conceito importante e difícil de definir, já que depende da interpretação do que seria uma região  em determinado caso.

Pelas figuras da seção \ref{sec:segmenthm} do capitulo \ref{cap:segmentacao}, nota-se as diferentes interpretações do conceito de região pelas diferentes quantidades de regiões notadas por humanos nas Figuras \ref{fig:Berkeley_mulher_segmentada} e \ref{fig:Berkeley_mulher_segmentada2}. Tal fato também acontece com os computadores, como nota-se pelas figuras \ref{fig:indiosegmentado}.

\subsubsection{Split and Merge}
\subsubsection*{Procedimento}
As etapas fundamentais deste algoritmo segundo  são: 
\begin{enumerate}
    \item Criar critério para definir o que é uma área homogênea.
    \item Começar com a imagem completa e divide em 4 sub-imagens.
    \item Checar cada sub-imagem e dividi-la novamente em 4 novas sub-imagens caso ela não seja homogênea.
    \item Repetir Passo 3 até que não se consiga mais subdividir.
    \item Comparar sub-imagens com suas regiões vizinhas e agrupá-las se forem homogêneas.
    \item Repetir Passo 5 até que não se consiga mais agrupar.
\end{enumerate}

\subsubsection*{EXEMPLO - QUADTREE}
Um exemplo de segmentação baseada em região Split and Merge é o algoritmo quadtree. A figura \ref{fig:aral} abaixo ilustra a segmentação de imagem por este algoritmo, e observa-se a segmentação da região que contém água na figura.

% ------------------------------------------------------------------------------------------------------------
% Figura 
\begin{figure}[!htb]
 \centering
 \def\baselinestretch{1}\small\normalsize
 \includegraphics[width=0.4\textwidth]{img/stf-aral1997.jpg}\qquad
 \includegraphics[width=0.4\textwidth]{img/stf-aral1997-quadtree.jpg} 
 \caption{\label{fig:aral}Uma representação visual de uma região \citep{stanford} à esquerda e à direita segmentada pelo algoritmo quadtree em linguagem de programação python.}
 %\vspace{2.0em}
\end{figure}

\subsubsection{Region growing (abordagem bottom-up)}
\subsubsection*{Procedimento}
As etapas fundamentais deste algoritmo são: 
\begin{enumerate}
    \item Identificar o ponto de partida.
    \item Incluir pixels vizinhos com características similares (nível de cinza, textura, cor, etc).
    \item Continuar até que todos os pixels estejam associados com um dos pontos de partida.
\end{enumerate}

\subsubsection*{EXEMPLO - WATERSHEED}
Um exemplo de segmentação baseada no método Region growing é o algoritmo watershed. A figura \ref{fig:coins} abaixo ilustra a segmentação de imagem por este algoritmo, e observa-se regiões distintas correspondentes à cada moeda da figura.

% ------------------------------------------------------------------------------------------------------------
% Figura 
\begin{figure}[!htb]
 \centering
 \def\baselinestretch{1}\small\normalsize
 \includegraphics[width=0.4\textwidth]{img/stf-coins.jpg}\qquad
 \includegraphics[width=0.4\textwidth]{img/stf-coins-watersheed.jpg} 
 \caption{\label{fig:coins}Imagem de uma moeda \citep{stanford} à esquerda e à direita segmentada pelo algoritmo watersheed em linguagem de programação python.}
 %\vspace{2.0em}
\end{figure}
 
% Stanford DataSet
%  https://scien.stanford.edu/index.php/test-images-and-videos/
 

