%%
%
% ARQUIVO: pre-texto.tex
%
% VERSÃO: 1.0
% DATA: Maio de 2016
% AUTOR: Coordenação de Trabalhos Especiais SE/8
% 
%  Arquivo tex para a criação da parte pré-textual do documento de Projeto de Fim de Curso.
%
%%


% -----
% PÁGINA DE CAPA DO DOCUMENTO DE PFC
% -----
\makecapa

% -----
% PÁGINA DE TÍTULO DO PFC
% -----
\prepareadvisors
\maketitle

% -----
% PÁGINA DE CRÉDITOS DO DOCUMENTO DE PFC
% -----
\makecredits

% -----
% PÁGINA DE FOLHA DE ASSINATURAS
% -----
\preparemembers
\approvalpage

% -----
% PÁGINA DE DEDICATÓRIA (OPCIONAL, ie. pode remover toda a página)
% -----
%%% DEDICATÓRIA - PREENCHER...
% -----
% PÁGINA DE DEDICATÓRIA (OPCIONAL, ie. pode remover toda a página)
% -----
%%% DEDICATÓRIA - PREENCHER...
\dedicatoria{%
Dedico aos meus familiares, amigos, minha esposa Mayra e professores do Instituto Militar de Engenharia, pois me deram forças e foram fundamentais para que eu pudesse vencer os obstáculos diários. 
Dedico aos meus pais e familiares pelo amor incondicional e pelos valores e ensinamentos passados que me fazem crescer e evoluir constantemente.
}%
\makededication

% -----
% PÁGINA DE AGRADECIMENTOS (OPCIONAL, ie. pode remover toda a página)
% -----
%%% AGRADECIMENTOS - PREENCHER...
\agradecimentos{%
Agradeço aos que estiveram comigo durante essa árdua jornada de estudos e me auxiliaram nessa etapa de desenvolvimento profissional. 
Um muito obrigado aos familiares, amigos e mestres.\\
\indent
Um obrigado especial aos Professores Orientadores Ph.D. Carla Liberal Pagliari e Ph.D. Marcelo de Mello Perez, por suas disponibilidades e atenções.
}%
\makethanks

% -----
% PÁGINA DE EPÍGRAFE (OPCIONAL, ie. pode remover toda a página)
% -----
%%% EPÍGRAFE - PREENCHER...
\epigrafe{%
Computação não se relaciona mais a computadores. Relaciona-se a viver.%
}%
\autorepigrafe{%    %% Se não tem autor, coloque "Anônimo"
Nicholas Negroponte%
}%
\makeepigraph

% -----
% PÁGINA DE SUMÁRIO
% -----
\tableofcontents

% -----
% PÁGINAS DE LISTAS DE FIGURAS E DE TABELAS
% se a Dissertação não possui figuras e/ou tabelas, REMOVA O COMANDO CORRESPONDENTE
% -----
\listoffigures
\listoftables

% -----
% PÁGINA DE LISTA DE SIGLAS
% se a Dissertação não possui siglas, REMOVA TODA A PÁGINA
% -----
%%% SIGLAS - PREENCHER...
% \acronimo{LA}{Los Angeles}


%  \listofnicks

% -----
% PÁGINA DE LISTA DE ABREVIATURAS
% se a Dissertação não possui abreviaturas ou símbolos, REMOVA TODA A PÁGINA
% -----
%%% ABREVIATURAS - PREENCHER...
\abreviatura{T}{Threshold}
\abreviatura{fij}{píxel na posição (i,j) }

%%% SÍMBOLOS - PREENCHER...
\simbolo{$\Phi$}{termo de dissipação visco	sa}
\simbolo{$\Gamma$}{coeficiente de difusão efetivo}
\simbolo{$\alpha$}{fator de sub-relaxação}
\simbolo{$\phi$}{variável dependente da equação diferencial geral}

% \listofsymbols

% -----
% PÁGINA DE RESUMO
% -----
%%% RESUMO - PREENCHER...
\resumo{%
Segmentação é uma importante etapa do processamento digital de imagens. Com um número cada vez maior de aplicações relacionadas ao domíno de visão computacional, mais presente se faz a necessidade de se aprofundar na identificação e análise de regiões relevantes de uma imagem a fim de obter melhores resultados e conclusões por meio da segmentação.\\
\indent 
Embora a identificação de imagens seja algo natural e uma atividade consideravelmente fácil para os seres humanos, a mesma tarefa em sistemas computacionais automatizados torna-se significativamente mais complexa. Com a crescente tendência de utilização de máquinas e sistemas para a realização das atividades na sociedade, nota-se a importância de desenvolver e implementar novas ferramentas capazes de trazer uma nova abordagem ao problema.   
Considerado como um problema mal colocado, isto é, para o qual não há uma solução universal, o estudo das principais técnicas consolidadas e o desenvolvimento de novas é a estratégia que se procura utilizar em projetos de pesquisa nesse domínio.\\
\indent
Esse projeto introduz o problema da segmentação, abordando sua motivação, suas dificuldades e sua aplicação, apresenta o estudo das técnicas mais relevantes e, como objetivo principal, desenvolve um aplicativo móvel em ambiente Android para a segmentação de imagens. Além do aprendizado de grande relevância obtido nessa importante área da tecnologia, esse projeto deixa uma contribuição positiva para a comunidade científica.
}%
\makeresumo

% -----
% PÁGINA DE ABSTRACT
% -----
%%% ABSTRACT - PREENCHER...
\abstract{%
Image segmentation is an important stage of the digital processing of images. With an increasing number of applications related to the field of computer vision, the more present the need to deepen the identification and analysis of relevant regions of an image in order to obtain better results and conclusions through segmentation.\\
\indent
Although the identification of images is something natural and a considerably easy activity for humans, the same task in automated computer systems becomes significantly more complex. With the growing tendency to use machines and systems to carry out activities in society, it is important to develop and implement new tools capable of bringing a new approach to the problem. Considered as an "ill-posed" problem, that is, for which there is no universal solution, the study of the main consolidated techniques and the development of new ones is the strategy that is sought to be used in research projects in this field.\\
\indent
This project introduces the problem of segmentation, approaching its motivation, difficulties and its application, presents the study of the most relevant techniques and, as main objective, develops an Android mobile application for the segmentation of images. In addition to the highly relevant learning gained in this important area of technology, this project leaves a positive contribution to the scientific community.
}%
\makeabstract
